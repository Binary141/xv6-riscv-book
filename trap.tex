%    Sidebar about panic:
% 	panic is the kernel's last resort: the impossible has happened and the
% 	kernel does not know how to proceed.  In xv6, panic does ...
\chapter{Traps and system calls}
\label{CH:TRAP}

There are three kinds of event which cause the CPU to set
aside ordinary execution of instructions and force a
transfer of control to special code that handles the event. One
situation is a system call, when a user program 
executes the {\tt ecall} instruction to ask the kernel to do 
something for it. Another situation is an \indextext{exception}:
an instruction (user or kernel) does something illegal, such as divide
by zero or use an invalid virtual address. The third situation is a
device \indextext{interrupt}, when a device signals that it needs
attention, for example when the disk hardware finishes a read or write
request.

This book uses \indextext{trap} as a generic term for these
situations. Typically whatever code was executing at the time of the
trap will later need to resume, and shouldn't need to be aware that
anything special happened. That is, we often want traps to be
transparent; this is particularly important for device interrupts, which the
interrupted code typically doesn't expect. The usual sequence is that
a trap forces a transfer of control into the kernel; the kernel saves
registers and other state so that execution can be resumed; the kernel
executes appropriate handler code (e.g., a system call implementation
or device driver); the kernel restores the saved state and returns
from the trap; and the original code resumes where it left off.

Xv6 handles all traps in the kernel; traps are not delivered to user
code. Handling traps in the kernel is natural for system calls. It
makes sense for interrupts since isolation demands that only the
kernel be allowed to use devices,
and because the kernel is a convenient mechanism with which to
share devices among multiple processes.
It also makes sense for exceptions
since xv6 responds to all exceptions from user space by killing the
offending program.

Xv6 trap handling proceeds in four stages: hardware actions taken by
the RISC-V CPU, an assembly ``vector'' that prepares the way for
kernel C code, a C trap handler that decides what to do with the trap,
and the system call or device-driver service routine. While
commonality among the three trap types suggests that a kernel could
handle all traps with a single code path, it turns out to be
convenient to have separate assembly vectors and C trap handlers for
three distinct cases: traps from user space, traps from kernel space,
and timer interrupts.

\section{RISC-V trap machinery}

Each RISC-V CPU has a set of control registers that the kernel writes to
tell the CPU how to handle traps, and that the kernel can read
to find out about a trap that has occured. The RISC-V documents
contain the full story~\cite{riscv:priv}. {\tt riscv.h}
\lineref{kernel/riscv.h:1} contains definitions that xv6 uses. Here's
an outline of the most important registers:

\begin{itemize}

\item \indexcode{stvec}: The kernel writes the address of its trap handler
  here; the RISC-V jumps to the address in {\tt stvec} to handle a trap.

\item \indexcode{sepc}: When a trap occurs, RISC-V saves the program counter
  here (since the {\tt pc} is then overwritten with the
  value in {\tt stvec}). The
  {\tt sret} (return from trap) instruction copies {\tt sepc} to the
  {\tt pc}. The kernel can write {\tt sepc} to control where {\tt
    sret} goes.

\item \indexcode{scause}: RISC-V puts a number here that describes
the reason for the trap.

\item \indexcode{sscratch}: The kernel places a value here that comes in
  handy at the very start of a trap handler.

\item \indexcode{sstatus}: The SIE bit in {\tt sstatus}
  controls whether device interrupts
  are enabled. If the kernel clears SIE, the RISC-V will defer
  device interrupts until the kernel sets SIE. The SPP bit
  indicates whether a trap came from user mode or supervisor
  mode, and controls to what mode {\tt sret} returns.

\end{itemize}

The above registers relate to traps handled in supervisor mode, and they
cannot be read or written in user mode. There is a similar set of
control registers for traps handled in machine mode; xv6 uses
them only for the special case of timer interrupts.

Each CPU on a multi-core chip has its own set of these registers,
and more than one CPU may be handling a trap at any given time.

When it needs to force a trap, the RISC-V hardware does the
following for all trap types (other than timer interrupts):

\begin{enumerate}

\item If the trap is a device interrupt, and the {\tt sstatus} SIE bit
  is clear, don't do any of the following.

\item Disable interrupts by clearing the SIE bit in {\tt sstatus}.

\item Copy the {\tt pc} to {\tt sepc}.

\item Save the current mode (user or supervisor) in the SPP bit in {\tt sstatus}.

\item Set {\tt scause} to reflect the trap's cause.

\item Set the mode to supervisor.

\item Copy {\tt stvec} to the {\tt pc}.

\item Start executing at the new {\tt pc}.

\end{enumerate}

Note that the CPU doesn't switch to the kernel page table, doesn't
switch to a stack in the kernel, and doesn't save any registers other
than the {\tt pc}. Kernel software must perform these tasks.
One reason that the CPU does minimal work during a trap is to provide
flexibility to software; for example, some operating systems 
omit a page table switch in some situations to increase
trap performance.

It's worth thinking about whether any of the steps listed above could
be omitted, perhaps in search of faster traps. Though there are
situations in which a simpler sequence can work, many of the steps
would be dangerous to omit in general. For example, suppose that the
CPU didn't switch program counters. Then a trap from user space could
switch to supervisor mode while still running user instructions. Those
user instructions could break user/kernel isolation, for example
by modifying the {\tt satp} register to point to a page table that
allowed accessing all of physical memory. It is thus important that
the CPU switch to a kernel-specified instruction address, namely {\tt
  stvec}.

\section{Traps from user space}

Xv6 handles traps differently depending on whether
it is executing in the kernel or in user code. Here is the
story for traps from user code; Section~\ref{s:ktraps}
describes traps from kernel code.

A trap may occur while executing in user space if the
user program makes a
system call ({\tt ecall} instruction), or does something
illegal, or if a device interrupts.
The high-level path of a trap from user space is
{\tt uservec}
\lineref{kernel/trampoline.S:/^uservec/},
then {\tt usertrap}
\lineref{kernel/trap.c:/^usertrap/};
and when returning,
{\tt usertrapret}
\lineref{kernel/trap.c:/^usertrapret/}
and then
{\tt userret}
\lineref{kernel/trampoline.S:/^uservec/}.

% talk about why RISC-V doesn't switch page tables

A major constraint on the design of xv6's trap handling is the fact
that the RISC-V hardware does not switch page tables when it forces a
trap. This means that the trap handler
address in {\tt stvec} must have a valid
mapping in the user page table, since that's the page table in force
when the trap handling code starts executing. Furthermore, xv6's trap
handling code needs to switch to the kernel page table; in order to be
able to continue executing after that switch, the kernel page table
must also have a mapping for the handler pointed to by {\tt stvec}.

Xv6 satisfies these requirements using a \indextext{trampoline} page.
The trampoline page contains {\tt uservec}, the xv6 trap handling code
that {\tt stvec} points to. The trampoline page is mapped in
every process's page table at address
\indexcode{TRAMPOLINE},
which is at the end of the virtual address space so that it will
be above memory that programs use for themselves.
The trampoline page is also mapped at address {\tt TRAMPOLINE}
in the kernel page table. See Figure~\ref{fig:as} and
Figure~\ref{fig:xv6_layout}. Because the trampoline page is mapped in
the user page table, with the {\tt PTE\_U} flag, traps can start
executing there in supervisor mode. Because the trampoline page is
mapped at the same address in the kernel address space, the trap handler
can continue to execute after it switches to the kernel page
table.

The code for the {\tt uservec} trap handler is in {\tt trampoline.S}
\lineref{kernel/trampoline.S:/^uservec/}.
When {\tt uservec} starts, all 32 registers contain values owned by
the interrupted user code. These 32 values need to be saved somewhere
in memory, so that
they can be restored when the trap returns to user space.
Storing to memory requires use of a register
to hold the address,
but at this point there are no general-purpose registers available!
Luckily RISC-V provides a helping hand in the
form of the {\tt sscratch} register. The {\tt csrrw} instruction at
the start of {\tt uservec} swaps the contents of {\tt a0} and {\tt
  sscratch}. Now the user code's {\tt a0} is saved
in {\tt sscratch}; {\tt uservec} has
one register ({\tt a0}) to play with; and {\tt a0} contains the
value the kernel previously placed in {\tt sscratch}.

{\tt uservec}'s next task is to save the 32 user registers. Before
entering user space, the kernel set {\tt sscratch} to point to a
per-process {\tt trapframe} structure that (among other things) has space to
save the 32 user registers
\lineref{kernel/proc.h:/^struct.trapframe/}. Because {\tt satp} still
refers to the user page table, {\tt uservec} needs the trapframe to be
mapped in the user address space. When creating each process, xv6
allocates a page for the process's trapframe, and arranges for it
always to be mapped at user virtual address {\tt TRAPFRAME}, which is
just below {\tt TRAMPOLINE}. The process's {\tt p->trapframe} also
points to the
trapframe, though at its physical address so the kernel can use it
through the kernel page table.

Thus after swapping {\tt a0} and {\tt sscratch}, {\tt a0}
holds a pointer to the current process's trapframe.
{\tt uservec} now saves all user registers there,
including the user's {\tt a0}, read from {\tt sscratch}.

The {\tt trapframe} contains the address of the current process's
kernel stack, the current CPU's hartid, the address of the {\tt usertrap}
function,
and the address of the kernel page table. {\tt uservec}
retrieves these values, switches {\tt satp} to the kernel page table,
and calls {\tt usertrap}.

The job of {\tt usertrap} is to determine
the cause of the trap, process it, and return
\lineref{kernel/trap.c:/^usertrap/}.
It first changes {\tt stvec} so
that a trap while in the kernel will be handled by
{\tt kernelvec} rather than {\tt uservec}.
It saves the {\tt sepc} (the saved user program counter),
because there might be a
process switch in {\tt usertrap} that could cause {\tt sepc}
to be overwritten.
If the trap is a system call, {\tt usertrap} calls {\tt syscall} to
handle it;
if a device interrupt, {\tt devintr};
otherwise it's an exception, and the kernel kills the
faulting process.
The system call path adds four to the saved user program counter
because RISC-V, in the case of a system call,
leaves the program pointer pointing to the {\tt ecall} instruction
but user code needs to resume executing at the subsequent instruction.
On the way out, {\tt usertrap} checks if the process has been
killed or should yield the CPU (if this trap is a timer interrupt).

The first step in returning to user space is the call to {\tt usertrapret}
\lineref{kernel/trap.c:/^usertrapret/}.
This function sets up the RISC-V control registers to prepare for a
future trap from user space. This involves changing {\tt stvec}
to refer to {\tt uservec}, preparing the trapframe fields that
{\tt uservec} relies on, and setting {\tt sepc} to the previously
saved user program counter. At the end, {\tt usertrapret}
calls {\tt userret} on the trampoline page that is mapped in
both user and kernel page tables; the reason is that assembly
code in {\tt userret} will switch page tables.

{\tt usertrapret}'s call to {\tt userret} passes a pointer to the process's user
page table in {\tt a0} and {\tt TRAPFRAME} in {\tt a1}
\lineref{kernel/trampoline.S:/^userret/}.
{\tt userret} switches {\tt satp} to the process's user page table.
Recall that the user page table maps both the trampoline page
and {\tt TRAPFRAME}, but nothing else from the kernel.
The fact that the trampoline page is mapped at the same
virtual address in user and kernel page tables is what allows
{\tt uservec} to keep executing after changing {\tt satp}.
{\tt userret} copies the trapframe's saved user {\tt a0} to {\tt sscratch}
in preparation for a later swap with TRAPFRAME.
From this point on, the only data {\tt userret} can use is
the register contents and the content of the trapframe.
Next {\tt userret} restores saved user registers from the trapframe,
does a final swap of {\tt a0} and {\tt sscratch} to restore the
user {\tt a0} and save {\tt TRAPFRAME} for the next trap,
and executes {\tt sret} to return to user space.

\section{Code: Calling system calls}

Chapter~\ref{CH:FIRST} ended with 
\indexcode{initcode.S}
invoking the {\tt exec} system call
\lineref{user/initcode.S:/SYS_exec/}.
Let's look at how the user call
makes its way to the {\tt exec} system call's
implementation in the kernel.

{\tt initcode.S}
places the arguments for
\indexcode{exec}
in registers {\tt a0} and {\tt a1}, and puts the
system call number in
\texttt{a7}.
System call numbers match the entries in the {\tt syscalls} array,
a table of function pointers
\lineref{kernel/syscall.c:/syscalls/}.
The \lstinline{ecall} instruction traps into the kernel
and causes
{\tt uservec},
{\tt usertrap}, and then {\tt syscall} to execute, as we saw above.

\indexcode{syscall}
\lineref{kernel/syscall.c:/^syscall/} 
retrieves the system call number from the saved
\texttt{a7} in the trapframe
and uses it to index into {\tt syscalls}.
For the first system call, 
\texttt{a7}
contains 
\indexcode{SYS_exec}
\lineref{kernel/syscall.h:/SYS_exec/},
resulting in a call to the system call implementation function
\lstinline{sys_exec}.

When \lstinline{sys_exec} returns,
\lstinline{syscall}
records its return value in
\lstinline{p->trapframe->a0}.
This will cause the original user-space call to 
{\tt exec()} to return that value, since the C
calling convention on RISC-V places return values in {\tt a0}.
System calls conventionally return negative numbers to indicate
errors, and zero or positive numbers for success.
If the system call number is invalid,
\lstinline{syscall}
prints an error and returns $-1$.

\section{Code: System call arguments}

System call implementations in the kernel need to find the arguments
passed by user code. Because user code calls system call wrapper
functions, the arguments are initially where the RISC-V C calling
convention places them: in registers.
The kernel trap code saves user registers to the current
process's trap frame, where kernel code can find them.
The kernel functions
\lstinline{argint},
\lstinline{argaddr},
and
\lstinline{argfd}
retrieve the 
\textit{n} 'th 
system call argument
from the trap frame
as an integer, pointer, or a file descriptor.
They all call {\tt argraw} to retrieve the appropriate saved
user register
\lineref{kernel/syscall.c:/^argraw/}.

Some system calls pass pointers as arguments, and the kernel must use
those pointers to read or write user memory. The {\tt exec} system
call, for example, passes the kernel an array of pointers
referring to string arguments in user space.
These pointers pose
two challenges. First, the user program may be buggy or malicious, and
may pass the kernel an invalid pointer or a pointer intended to trick
the kernel into accessing kernel memory instead of user memory.
Second, the xv6 kernel page table mappings are not the same as the
user page table mappings, so the kernel cannot use ordinary
instructions to load or store from user-supplied addresses.

The kernel implements functions that safely transfer data to and
from user-supplied addresses.
{\tt fetchstr} is an example \lineref{kernel/syscall.c:/^fetchstr/}.
File system calls such as
{\tt exec} use {\tt fetchstr} to retrieve string file-name arguments from user
space.
\lstinline{fetchstr} calls \lstinline{copyinstr}
to do the hard work.

\indexcode{copyinstr}
\lineref{kernel/vm.c:/^copyinstr/} copies up to \lstinline{max} bytes to
\lstinline{dst} from virtual address \lstinline{srcva} in the user page
table \lstinline{pagetable}.
Since \lstinline{pagetable} is {\it not} the current page
table,
\lstinline{copyinstr} uses {\tt walkaddr}
(which calls {\tt walk}) to look up
\lstinline{srcva} in
\lstinline{pagetable}, yielding
physical address \lstinline{pa0}.
The kernel maps each physical RAM address to the corresponding
kernel virtual address, so
{\tt copyinstr} can directly copy string bytes from {\tt pa0} to {\tt dst}.
{\tt walkaddr} 
\lineref{kernel/vm.c:/^walkaddr/}
checks that the user-supplied virtual address is part of
the process's user address space, so programs
cannot trick the kernel into reading other memory.
A similar function, {\tt copyout}, copies data from the
kernel to a user-supplied address.

\section{Traps from kernel space}
\label{s:ktraps}

Xv6 configures the CPU trap registers somewhat differently depending
on whether user or kernel code is executing.
When the kernel is executing on a CPU, the kernel points {\tt stvec}
to the assembly code at {\tt kernelvec}
\lineref{kernel/kernelvec.S:/^kernelvec/}.
Since xv6 is already in the kernel, {\tt kernelvec} can rely
on {\tt satp} being set to the kernel page table, and on the
stack pointer referring to a valid kernel stack.
{\tt kernelvec} pushes all 32 registers onto the stack,
from which it will later restore them
so that the interrupted
kernel code can resume without disturbance.

{\tt kernelvec} saves the registers on the stack of the interrupted
kernel thread, which makes sense because the register values belong to
that thread. This is particularly important if the trap causes a
switch to a different thread -- in that case the trap will actually
return from the stack of the new thread, leaving the interrupted
thread's saved registers safely on its stack.

{\tt kernelvec} jumps to {\tt kerneltrap}
\lineref{kernel/trap.c:/^kerneltrap/} after saving registers.
{\tt kerneltrap} is prepared for two types of traps:
device interrrupts and exceptions. It calls
{\tt devintr}
\lineref{kernel/trap.c:/^devintr/}
to check for and handle the former.
If the trap isn't a device interrupt, it must be an exception,
and that is always a fatal error if it occurs in the xv6 kernel;
the kernel calls \lstinline{panic} and stops executing.

If {\tt kerneltrap} was called due to a timer interrupt, and a
process's kernel thread is running (as opposed to a scheduler thread),
{\tt kerneltrap} calls {\tt yield} to give other threads a chance to
run. At some point one of those threads will yield, and let our thread
and its {\tt kerneltrap} resume again.
Chapter~\ref{CH:SCHED} explains what happens in {\tt yield}.

When {\tt kerneltrap}'s work is done, it needs to return to whatever
code was interrupted by the trap. Because a {\tt yield} may have
disturbed the saved {\tt sepc} and the saved previous mode in {\tt sstatus},
{\tt kerneltrap} saves them when it starts. It now restores those
control registers and returns to {\tt kernelvec}
\lineref{kernel/kernelvec.S:/call.kerneltrap$/}.
{\tt kernelvec} pops the saved registers from the stack and
executes {\tt sret}, which copies {\tt sepc} to {\tt pc}
and resumes the interrupted kernel code.

It's worth thinking through how the trap return happens if
{\tt kerneltrap} called {\tt yield} due to a timer interrupt.

Xv6 sets a CPU's {\tt stvec} to {\tt kernelvec} when that CPU
enters the kernel from user space; you can see this in {\tt usertrap}
\lineref{kernel/trap.c:/stvec.*kernelvec/}.
There's a window of time when the kernel has started executing
but {\tt stvec} is still set to {\tt uservec}, and it's crucial that 
no device interrupt occur during that window.
Luckily the RISC-V always disables interrupts when it starts
to take a trap, and xv6 doesn't enable them again until
after it sets {\tt stvec}.

\section{Page-fault exceptions}
\label{sec:pagefaults}

Xv6's response to exceptions is quite boring: if an exception happens
in user space, the kernel kills the faulting process.  If an
exception happens in the kernel, the kernel panics.  Real operating systems
often respond in much more interesting ways.

As an example,
many kernels use page faults to implement
\indextext{copy-on-write (COW) fork}.
To explain copy-on-write fork, consider xv6's \lstinline{fork},
described in Chapter~\ref{CH:MEM}.
\lstinline{fork} causes the child to have the same
memory content as the parent, by calling
\lstinline{uvmcopy}
\lineref{kernel/vm.c:/^uvmcopy/}
to allocate physical memory for
the child and copy the parent's memory into it.
It would be more efficient if the child and parent could share 
the parent's physical memory.
A straightforward implementation of this would not work, however,
since it would cause the parent and child to disrupt each other's
execution with their writes to the shared stack and heap.

Parent and child can safely share physical memory by
appropriate use of page-table permissions and page faults.
The CPU raises a
\indextext{page-fault exception}
when a virtual address is used that has no mapping
in the page table, or a mapping whose \lstinline{PTE_V}
flag is clear, or a mapping whose permission bits
(\lstinline{PTE_R},
\lstinline{PTE_W},
\lstinline{PTE_X},
\lstinline{PTE_U})
forbid the operation being attempted.
RISC-V distinguishes three
kinds of page fault: load page faults (when a load instruction cannot
translate its virtual address), store page faults (when a store
instruction cannot translate its virtual address), and instruction
page faults (when the address for an instruction doesn't translate).  The
value in the \lstinline{scause} register indicates the type of the
page fault and the \indexcode{stval} register contains the address
that couldn't be translated.

The basic plan in COW fork is for the parent and child to initially share all
physical pages, but to map them read-only (with the
\lstinline{PTE_W} flag clear).
Thus, when the child or parent executes a
store instruction, the RISC-V CPU raises a page-fault exception. In
response to this exception, the kernel makes a copy of the page that
contains the faulted address. It maps one copy read/write in the
child's address space and the other copy read/write in the parent's address
space.  After updating the page tables, the kernel resumes the
faulting process at the instruction that caused the fault. Because the
kernel has updated the relevant PTE to allow writes,
the faulting instruction will now
execute without a fault.

Copy-on-write makes \lstinline{fork} faster, since \lstinline{fork}
need not copy memory. Some of the memory will have to be copied
later, when written, but it's often the case that most of the
memory need never be copied.
A common example is
\lstinline{fork} followed by \lstinline{exec}:
a few pages of stack may be written between the
two system calls, but then the \lstinline{exec} frees
most of the child memory inherited from the parent.
Copy-on-write \lstinline{fork} eliminates the need to
copy this memory.
Furthermore, COW fork is transparent:
no modifications to applications are necessary for
them to benefit.

The combination of page tables and page faults opens up a wide range
of interesting possibilities in addition to COW fork.  Another widely-used
feature is called \indextext{lazy allocation}, which has two parts.
First, when an application asks for more memory by
calling \lstinline{sbrk}, the kernel 
notes the increase in size,
but does not allocate physical memory and does not create PTEs
for the new range of virtual addresses.
Second, on a page fault on one of those new addresses,
the kernel allocates a page of physical
memory and maps it into the page table.  Since applications often ask
for more memory than they need, lazy allocation is a win: the kernel
allocates memory only when the application actually uses it.  Like
COW fork, the kernel can implement this feature transparently to
applications.

Yet another widely-used feature that exploits page faults
is \indextext{paging from disk}.  If applications need more
memory than the available physical RAM,
the kernel can evict some pages: write
them to a storage device such as a disk and mark their
PTEs as not valid. If an application reads
or writes an
evicted page, the CPU will experience a page fault. The kernel can
then inspect the faulting address. If the address belongs to a page
that is on disk, the kernel allocates a page of physical
memory,
reads the page from disk to that memory, updates
the PTE to be valid and refer to that memory,
and resumes the application. To make room for the page,
the kernel may have to evict another page.  This feature requires no
changes to applications, and works well if applications have locality
of reference (i.e., they use only a subset of their memory
at any given time).

Other features that combine paging and page-fault exceptions include
automatically extending stacks and memory-mapped files.

\section{Real world}

The need for special trampoline pages could be eliminated if kernel
memory were mapped into every process's user page table (with
appropriate PTE permission flags). That would
also eliminate the need for a page table switch when trapping from
user space into the kernel. That in turn would allow system call
implementations in the kernel to take advantage of the current
process's user memory being mapped, allowing kernel code to directly
dereference user pointers. Many operating systems have used these ideas to
increase efficiency. Xv6 avoids them in order to reduce the chances of
security bugs in the kernel due to inadvertent use of user pointers,
and to reduce some complexity that would be required to ensure that
user and kernel virtual addresses don't overlap.

\section{Exercises}

\begin{enumerate}

\item The functions {\tt copyin} and {\tt copyinstr} walk the user
  page table in software.  Set up the kernel page table so that the
  kernel has the user program mapped, and {\tt copyin} and {\tt
    copyinstr} can use {\tt memcpy} to copy system call arguments into
  kernel space, relying on the hardware to do the page table walk.

\item Implement lazy memory allocation.

\item Implement COW fork.

\item Is there a way to eliminate the special {\tt
  TRAPFRAME} page mapping in every user address space? For
  example, could
  {\tt uservec} be modified to simply push the 32 user registers
  onto the kernel stack, or store them in the {\tt proc}
  structure?

\item Could xv6 be modified to eliminate the special {\tt
  TRAMPOLINE} page mapping?

\end{enumerate}
