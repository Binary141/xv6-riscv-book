\chapter{Concurrency revisited}
\label{CH:LOCK2}

Simultaneously obtaining good parallel
performance, correctness despite concurrency, and understandable code
is a big challenge in kernel design.
Straightforward use of locks is the best path to correctness,
but is not always possible.
This
chapter highlights examples in which xv6 is forced to use locks in an
involved way, and examples where xv6 uses lock-like techniques but not
locks.

\section{Locking patterns}

Cached items are often a challenge to lock.
For example,
the filesystem's block cache \lineref{kernel/bio.c:/^struct/} stores
copies of up to {\tt NBUF} disk blocks.
It's vital that a given disk block have at most
one copy in the cache; otherwise, different processes might make
conflicting changes to different copies of what ought to be the same
block. Each cached block is stored in a {\tt struct buf}
\lineref{kernel/buf.h:1}. A {\tt struct buf} has a lock field which
helps ensure that only one process uses a given disk block at a time.
However, that lock is not enough: what if a block is not present in
the cache at all, and two processes want to use it at the same time?
There is no {\tt struct buf} (since the block isn't yet cached), and
thus there is nothing to lock. Xv6 deals with this situation by
associating an additional lock ({\tt bcache.lock}) with the set of
identities of cached blocks. Code that needs to check if a block is
cached (e.g., {\tt bget} \lineref{kernel/bio.c:/^bget/}), or change the
set of cached blocks, must hold {\tt bcache.lock}; after that code has
found the block and {\tt struct buf} it needs, it can release {\tt
  bcache.lock} and lock just the specific block. This is a common
pattern: one lock for the set of items, plus one lock per item.

Ordinarily the same function that acquires a lock will release it. But
a more precise way to view things is that a lock is acquired at the
start of a sequence that must appear atomic, and released when that
sequence ends. If the sequence starts and ends in different functions,
or different threads, or on different CPUs, then the lock acquire and
release must do the same. The function of the lock is to force
other uses to wait, not to pin a piece of data to a particular agent. One
example is the {\tt acquire} in {\tt yield}
\lineref{kernel/proc.c:/^yield/}, which is released in the scheduler
thread rather than in the acquiring process. Another example is the
{\tt acquiresleep} in {\tt ilock} \lineref{kernel/fs.c:/^ilock/}; this
code often sleeps while reading the disk; it may wake up on a
different CPU, which means the lock may be acquired
and released on different CPUs.

Freeing an object that is protected by a lock embedded in the object
is a delicate business, since owning the lock is
not enough to guarantee that freeing would be correct. The problem
case arises when some other thread is waiting in {\tt acquire} to use
the object; freeing the object implicitly frees the embedded lock, which will
cause the waiting thread to malfunction. One solution is to track how
many references to the object exist, so that it is only freed when the
last reference disappears. See {\tt pipeclose}
\lineref{kernel/pipe.c:/^pipeclose/} for an example;
{\tt pi->readopen} and {\tt pi->writeopen} track whether
the pipe has file descriptors referring to it.

% sleep locks.
% hand-over-hand locking in namei.
%   namei's use of refcount to prevent changing underfoot,
%     and lock when actually using it.
% example of where deadlock is avoided? namei?
% spawn a thread to evade a lock order problem?

\section{Lock-like patterns}

In many places xv6 uses a reference count or a flag in a lock-like way
to indicate that an object is allocated and should not be freed
or re-used. A process's {\tt p->state} acts in this way, as do the
reference counts in {\tt file}, {\tt inode}, and {\tt buf} structures.
While in each case a lock protects the flag or reference count, it is
the latter that prevents the object from being prematurely freed.

The file system uses {\tt struct inode} reference counts as a kind of
shared lock that can be held by multiple processes, in order to avoid
deadlocks that would occur if the code used ordinary locks. For
example, the loop in {\tt namex} \lineref{kernel/fs.c:/^namex/} locks
the directory named by each pathname component in turn. However, {\tt namex}
must release each lock at the end of the loop, since if it
held multiple locks it could deadlock with itself if the pathname
included a dot (e.g., {\tt a/./b}). It might also deadlock with a
concurrent lookup involving the directory and {\tt ..}. As
Chapter~\ref{CH:FS} explains, the solution is for the loop to carry
the directory inode over to the next iteration with its reference
count incremented, but not locked.

Some data items are protected by different mechanisms at different
times, and may at times be protected from concurrent access implicitly
by the structure of the xv6 code rather than by explicit locks. For
example, when a physical page is free, it is protected by \texttt{kmem.lock}
\lineref{kernel/kalloc.c:/^. kmem;/}. If the page is then
allocated as a pipe \lineref{kernel/pipe.c:/^pipealloc/}, it is
protected by a different lock (the embedded \lstinline{pi->lock}). If the page
is re-allocated for a new process's user memory, it is not protected by a
lock at all. Instead, the fact that the allocator won't give that page
to any other process (until it is freed) protects it from concurrent
access. 
The ownership of a new process's memory is complex:
first the parent allocates and
manipulates it in {\tt fork}, then the child uses it, and (after the
child exits) the parent again owns the memory and passes it to {\tt
  kfree}. There are two lessons here: a data object may be protected
from concurrency in different ways at different points in its
lifetime, and the protection may take the form of implicit structure
rather than explicit locks.

A final lock-like example is the need to disable interrupts around
calls to {\tt mycpu()} \lineref{kernel/proc.c:/^myproc/}. Disabling
interrupts causes the calling code to be atomic with respect to timer
interrupts that could force a context switch, and thus move the
process to a different CPU.

\section{No locks at all}

There are a few places where xv6 shares mutable data with no locks at
all. One is in the implementation of spinlocks, although one could
view the RISC-V atomic instructions as relying on locks implemented in
hardware. Another is the {\tt started} variable in {\tt main.c}
\lineref{kernel/main.c:/^volatile/}, used to prevent other CPUs from
running until CPU zero has finished initializing xv6;
the {\tt volatile} ensures that the compiler actually generates
load and store instructions.

Xv6 contains cases in which one CPU or thread writes some data, and
another CPU or thread reads the data, but there is no specific lock
dedicated to protecting that data. For example, in {\tt fork}, the
parent writes the child's user memory pages, and the child (a
different thread, perhaps on a different CPU) reads those pages; no
lock explicitly protects those pages. This is not strictly a locking
problem, since the child doesn't start executing until after the parent has
finished writing. It is a potential memory ordering problem
(see Chapter~\ref{CH:LOCK}), since without a memory barrier there's no
reason to expect one CPU to see another CPU's writes. However, since
the parent releases locks, and the child acquires locks as it starts
up, the memory barriers in {\tt acquire} and {\tt release}
ensure that the child's CPU sees the parent's writes.

\section{Parallelism}

Locking is primarily about suppressing parallelism in the interests of
correctness. Because performance is also important, kernel designers
often have to think about how to use locks in a way that both achieves 
correctness and allows parallelism. While xv6 is not systematically
designed for high performance, it's still worth considering which xv6
operations can execute in parallel, and which might conflict on locks.

Pipes in xv6 are an example of fairly good parallelism. Each pipe has
its own lock, so that different processes can read and write 
different pipes in parallel on different CPUs. For a given pipe,
however, the writer and reader must wait for each other to release the
lock; they can't read/write the same pipe at the same time. It is also
the case that a read from an empty pipe (or a write to a full pipe)
must block, but this is not due to the locking scheme.

Context switching is a more complex example. Two kernel threads, each
executing on its own CPU, can call {\tt yield}, {\tt sched}, and {\tt
  swtch} at the same time, and the calls will execute in parallel. The
threads each hold a lock, but they are different locks, so they don't
have to wait for each other. Once in {\tt scheduler}, however, the two
CPUs may conflict on locks while searching the table of processes for
one that is {\tt RUNNABLE}. That is, xv6 is likely to get a
performance benefit from multiple CPUs during context switch, but
perhaps not as much as it could.

Another example is concurrent calls to {\tt fork} from different
processes on different CPUs. The calls may have to wait for each other
for {\tt pid\_lock} and {\tt kmem.lock}, and for per-process locks
needed to search the process table for an {\tt UNUSED} process. On the
other hand, the two forking processes can copy user memory pages and
format page-table pages fully in parallel.

The locking scheme in each of the above examples sacrifices parallel
performance in certain cases. In each case it's possible to obtain
more parallelism using a more elaborate design. Whether it's
worthwhile depends on details: how often the relevant operations are
invoked, how long the code spends with a contended lock held, how many
CPUs might be running conflicting operations at the same time, whether
other parts of the code are more restrictive bottlenecks. It can be
difficult to guess whether a given locking scheme might cause
performance problems, or whether a new design is significantly better,
so measurement on realistic workloads is often required.

\section{Exercises}

\begin{enumerate}

\item Modify xv6's pipe implementation to allow a read and
  a write to the same pipe to proceed in parallel on different cores.

\item Modify xv6's \texttt{scheduler()} to reduce lock contention
  when different cores are looking for runnable processes at the same time.

\item Eliminate some of the serialization in xv6's \texttt{fork()}.

\end{enumerate}
